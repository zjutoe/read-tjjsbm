\chapter{通鉴纪事本末·卷一}
\label{chap:vol1}

\section{三家分晉}
\label{sec:san-jia-fen-jin}

\todo[inline,color=green!30] {
1.处理器体系结构发展的历史和趋势:从单核时代发掘ILP,到多核时代需要发掘TLP\\
2.并行度不足的挑战:G. Blake等\cite{blake2010evolution}\\
3.MIT Seven-Dwarfs纲领\cite{asanovic2006landscape}\\
}


周威烈王二十三年,初命晉大夫魏斯、趙籍、韓虔爲諸侯。

    臣光曰:臣聞天子之職莫大於禮,禮莫大於分,分莫大於名。何謂禮?紀綱是也。何謂分?君、臣是也。何謂名?公、侯、卿、大夫是也。

    夫以四海之廣,兆民之衆,受制於一人,雖有絕倫之力,高世之智,莫不奔走而服役者,豈非以禮為之紀綱哉!是故天子統三公,三公率諸侯,諸侯制卿大夫,卿大夫治士庶人。貴以臨賤,賤以承貴。上之使下猶心腹之運手足,根本之制支葉,下之事上猶手足之衞心腹,支葉之庇本根,然後能上下相保而國家治安。故曰天子之職莫大於禮也。

    文王序易,以乾、坤為首。孔子繫之曰:「天尊地卑,乾坤定矣。卑高以陳,貴賤位矣。」言君臣之位猶天地之不可易也。春秋抑諸侯,尊王室,王人雖微,序於諸侯之上,以是見聖人於君臣之際未嘗不惓惓也。非有桀、紂之暴,湯、武之仁,人歸之,天命之,君臣之分當守節伏死而已矣。是故以微子而代紂則成湯配天矣,以季札而君吳則太伯血食矣,然二子寧亡國而不為者,誠以禮之大節不可亂也。故曰禮莫大於分也。

    夫禮,辨貴賤,序親疏,裁羣物,制庶事,非名不著,非器不形。名以命之,器以別之,然後上下粲然有倫,此禮之大經也。名器旣亡,則禮安得獨在哉!昔仲叔於奚有功於衞,辭邑而請繁纓,孔子以為不如多與之邑。惟名與器,不可以假人,君之所司也;政亡則國家從之。衞君待孔子而為政,孔子欲先正名,以為名不正則民無所措手足。夫繁纓,小物也,而孔子惜之;正名,細務也,而孔子先之:誠以名器旣亂則上下無以相保故也。夫事未有不生於微而成於著,聖人之慮遠,故能謹其微而治之,衆人之識近,故必待其著而後救之;治其微則用力寡而功多,救其著則竭力而不能及也。易曰:「履霜堅冰至,」書曰:「一日二日萬幾,」謂此類也。故曰分莫大於名也。

    嗚呼!幽、厲失德,周道日衰,綱紀散壞,下陵上替,諸侯專征,大夫擅政,禮之大體什喪七八矣,然文、武之祀猶緜緜相屬者,蓋以周之子孫尚能守其名分故也。何以言之?昔晉文公有大功於王室,請隧於襄王,襄王不許,曰:「王章也。未有代德而有二王,亦叔父之所惡也。不然,叔父有地而隧,又何請焉!」文公於是懼而不敢違。是故以周之地則不大於曹、滕,以周之民則不衆於邾、莒,然歷數百年,宗主天下,雖以晉、楚、齊、秦之強不敢加者,何哉?徒以名分尚存故也。至於季氏之於魯,田常之於齊,白公之於楚,智伯之於晉,其勢皆足以逐君而自為,然而卒不敢者,豈其力不足而心不忍哉,乃畏奸名犯分而天下共誅之也。今晉大夫暴蔑其君,剖分晉國,天子旣不能討,又寵秩之,使列於諸侯,是區區之名分復不能守而幷棄之也。先王之禮於斯盡矣。

    或者以為當是之時,周室微弱,三晉強盛,雖欲勿許,其可得乎!是大不然。夫三晉雖強,苟不顧天下之誅而犯義侵禮,則不請於天子而自立矣。不請於天子而自立,則為悖逆之臣,天下苟有桓、文之君,必奉禮義而征之。今請於天子而天子許之,是受天子之命而為諸侯也,誰得而討之!故三晉之列於諸侯,非三晉之壞禮,乃天子自壞之也。

    嗚呼!君臣之禮旣壞矣,則天下以智力相雄長,遂使聖賢之後為諸侯者,社稷無不泯絕,生民之類糜滅幾盡,豈不哀哉!

初,智宣子將以瑤為後,智果曰:「不如宵也。瑤之賢於人者五,其不逮者一也。美鬢長大則賢,射御足力則賢,伎藝畢給則賢,巧文辯惠則賢,強毅果敢則賢;如是而甚不仁。夫以其五賢陵人而以不仁行之,其誰能待之?若果立瑤也,智宗必滅。」弗聽。智果別族於太史,為輔氏。

趙簡子之子,長曰伯魯,幼曰無恤。將置後,不知所立,乃書訓戒之辭於二簡,以授二子曰:「謹識之!」三年而問之,伯魯不能舉其辭;求其簡,已失之矣。問無恤,誦其辭甚習;求其簡,出諸袖中而奏之。於是簡子以無恤為賢,立以為後。

簡子使尹鐸為晉陽,請曰:「以為繭絲乎?抑為保障乎?」簡子曰:「保障哉!」尹鐸損其戶數。簡子謂無恤曰:「晉國有難,而無以尹鐸為少,無以晉陽為遠,必以為歸。」

及智宣子卒,智襄子為政,與韓康子、魏桓子宴於藍臺。智伯戲康子而侮段規。智國聞之,諫曰:「主不備難,難必至矣!」智伯曰:「難將由我。我不為難,誰敢興之!」對曰:「不然。夏書有之:『一人三失,怨豈在明,不見是圖。』夫君子能勤小物,故無大患。今主一宴而恥人之君相,又弗備,曰『不敢興難』,無乃不可乎!蜹、蟻、蜂、蠆,皆能害人,況君相乎!」弗聽。

智伯請地於韓康子,康子欲弗與。段規曰:「智伯好利而愎,不與,將伐我;不如與之。彼狃於得地,必請於他人;他人不與,必嚮之以兵,然後我得免於患而待事之變矣。」康子曰:「善。」使使者致萬家之邑於智伯。智伯悅。又求地於魏桓子,桓子欲弗與。任章曰:「何故弗與?」桓子曰:「無故索地,故弗與。」任章曰:「無故索地,諸大夫必懼;吾與之地,智伯必驕。彼驕而輕敵,此懼而相親;以相親之兵待輕敵之人,智氏之命必不長矣。周書曰:『將欲敗之,必姑輔之。將欲取之,必姑與之。』主不如與之,以驕智伯,然後可以擇交而圖智氏矣,柰何獨以吾為智氏質乎!」桓子曰:「善。」復與之萬家之邑一。

智伯又求蔡、皋狼之地於趙襄子,襄子弗與。智伯怒,帥韓、魏之甲以攻趙氏。襄子將出,曰:「吾何走乎?」從者曰:「長子近,且城厚完。」襄子曰:「民罷力以完之,又斃死以守之,其誰與我!」從者曰:「邯鄲之倉庫實。」襄子曰:「浚民之膏澤以實之,又因而殺之,其誰與我!其晉陽乎,先主之所屬也,尹鐸之所寬也,民必和矣。」乃走晉陽。

三家以國人圍而灌之,城不浸者三版;沈竈產鼃,民無叛意。智伯行水,魏桓子御,韓康子驂乘。智伯曰:「吾乃今知水可以亡人國也。」桓子肘康子,康子履桓子之跗,以汾水可以灌安邑,絳水可以灌平陽也。絺疵謂智伯曰:「韓、魏必反矣。」智伯曰:「子何以知之?」絺疵曰:「以人事知之。夫從韓、魏之兵以攻趙,趙亡,難必及韓、魏矣。今約勝趙而三分其地,城不沒者三版,人馬相食,城降有日,而二子無喜志,有憂色,是非反而何?」明日,智伯以絺疵之言告二子,二子曰:「此夫讒人欲為趙氏游說,使主疑於二家而懈於攻趙氏也。不然,夫二家豈不利朝夕分趙氏之田,而欲為危難不可成之事乎!」二子出,絺疵入曰:「主何以臣之言告二子也?」智伯曰:「子何以知之?」對曰:「臣見其視臣端而趨疾,知臣得其情故也。」智伯不悛。絺疵請使於齊。

趙襄子使張孟談潛出見二子,曰:「臣聞脣亡則齒寒。今智伯帥韓、魏以攻趙,趙亡則韓、魏為之次矣。」二子曰:「我心知其然也;恐事末遂而謀泄,則禍立至矣。」張孟談曰:「謀出二主之口,入臣之耳,何傷也!」二子乃潛與張孟談約,為之期日而遣之。襄子夜使人殺守隄之吏,而決水灌智伯軍。智伯軍救水而亂,韓、魏翼而擊之,襄子將卒犯其前,大敗智伯之衆,遂殺智伯,盡滅智氏之族。唯輔果在。

    臣光曰:智伯之亡也,才勝德也。夫才與德異,而世俗莫之能辨,通謂之賢,此其所以失人也。夫聰察強毅之謂才,正直中和之謂德。才者,德之資也;德者,才之帥也。雲夢之竹,天下之勁也;然而不矯揉,不羽括,則不能以入堅。棠谿之金,天下之利也;然而不鎔範,不砥礪,則不能以擊強。是故才德全盡謂之「聖人」,才德兼亡謂之「愚人」;德勝才謂之「君子」,才勝德謂之「小人」。凡取人之術,苟不得聖人、君子而與之,與其得小人,不若得愚人。何則?君子挾才以為善,小人挾才以為惡。挾才以為善者,善無不至矣;挾才以為惡者,惡亦無不至矣。愚者雖欲為不善,智不能周,力不能勝,譬如乳狗搏人,人得而制之。小人智足以遂其姦,勇足以決其暴,是虎而翼者也,其為害豈不多哉!夫德者人之所嚴,而才者人之所愛;愛者易親,嚴者易疏,是以察者多蔽於才而遺於德。自古昔以來,國之亂臣,家之敗子,才有餘而德不足,以至於顛覆者多矣,豈特智伯哉!故為國為家者苟能審於才德之分而知所先後,又何失人之足患哉!

三家分智氏之田。趙襄子漆智伯之頭,以為飲器。智伯之臣豫讓欲為之報仇,乃詐為刑人,挾匕首,入襄子宮中塗廁。襄子如廁心動,索之,獲豫讓。左右欲殺之,襄子曰:「智伯死無後,而此人欲為報仇,真義士也,吾謹避之耳。」乃舍之。豫讓又漆身為癩,吞炭為啞。行乞於市,其妻不識也。行見其友,其友識之,為之泣曰:「以子之才,臣事趙孟,必得近幸。子乃為所欲為,顧不易邪?何乃自苦如此?求以報仇,不亦難乎!」豫讓曰:「旣已委質為臣,而又求殺之,是二心也。凡吾所為者,極難耳。然所以為此者,將以愧天下後世之為人臣懷二心者也。」襄子出,豫讓伏於橋下。襄子至橋,馬驚;索之,得豫讓,遂殺之。

襄子為伯魯之不立也,有子五人,不肯置後。封伯魯之子於代,曰代成君,早卒;立其子浣為趙氏後。襄子卒,弟桓子逐浣而自立;一年卒。趙氏之人曰:「桓子立非襄主意。」乃共殺其子,復迎浣而立之,是為獻子。獻子生籍,是為烈侯。魏斯者,魏桓子之孫也,是為文侯。韓康子生武子;武子生虔,是為景侯。

魏文侯以卜子夏、田子方為師。每過段干木之廬必式。四方賢士多歸之。

文侯與羣臣飲酒,樂,而天雨,命駕將適野。左右曰:「今日飲酒樂,天又雨,君將安之?」文侯曰:「吾與虞人期獵,雖樂,豈可無一會期哉!」乃往,身自罷之。

韓借師於魏以伐趙,文侯曰:「寡人與趙,兄弟也,不敢聞命。」趙借師於魏以伐韓,文侯應之亦然。二國皆怒而去。已而知文侯以講於己也,皆朝於魏。魏於是始大於三晉,諸侯莫能與之爭。 


